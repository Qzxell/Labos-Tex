\section{7. Conclusiones}
\begin{itemize}
  \item Respecto al resorte, tenemos un comportamiento elástico debido a la relación presentada en las gráficas de las figuras 8 y 9 .
  \item Se pudo establecer una relación entre esfuerzo real y deformación unitaria por tracción, por lo que el objetivo del experimento ha sido cumplido y los valores de la constante elástica del resorte y módulo de Young han podido ser calculados.
  \item Respecto al resorte, se puede notar que la deformación unitaria es mayor en las cargas 5 y 6 de la tabla 2, por lo que la longitud final del resorte fue mayor al doble de la longitud natural del mismo.
  \item Se observa el efecto de las deformaciones transversales en el resorte, llegando a reducir el diámetro de la sección transversal del resorte. Esto quiere decir que, a mayor esfuerzo, hay mayor tracción y por lo tanto, la compresión transversal sera mayor también, reduciendo el área de la sección transversal al reducir el diámetro de esta.
  \item Respecto a la liga, se observa un comportamiento elástico muy diferente al del resorte, llegando a recuperar su forma después de un largo tiempo.
  \item Observamos en la tabla 3 y 4, los valores difieren; esto ocurre por lo mencionado en el ítem anterior.
  \item Respecto a la gráfica esfuerzo-deformación (figura 10) de la liga, podemos observar un comportamiento similar al ocurrido en la histéresis elástica.
  \item De la gráfica en la figura 10, podemos notar que los tres primeros puntos del proceso de carga parecen estar alineados(cumpliendo una relación proporcional), pero no sería certero elegir dichos puntos y conformar una zona elástica, ya que la curva de descarga no pasa por dichos puntos.
  \item El no considerar el peso de la liga y resorte fue útil para un cálculo rápido. Cabe resaltar también, que las longitudes naturales medidas incluyen a la deformación producto del peso del resorte o liga.
\end{itemize}
