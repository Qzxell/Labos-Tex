\section{6. Cuestionario, observaciones y sugerencias}
\subsection{6.1. Cuestionario}
\begin{enumerate}
  \item Llenar la tabla dada por el profesor en clase(con incertidumbres).
\end{enumerate}

R: Tablas $1,2,3$ y 4 en las páginas 12 y 13 . Se adjunta también en la parte final del documento para mantener un orden.\\
2) Para el resorte, hacer las siguientes gráficas:\\
a) Peso vs $\Delta \ell$.

R: Gráfica de la figura 8 , página 13.\\
b) $\sigma$ vs $\varepsilon$ (esfuerzo real vs deformación unitaria).

R: Gráfica de la figura 9, página 14.\\
En cada gráfico, describir la relación existente entre las magnitudes.\\
R: De la figura 8 , tenemos una relación aproximadamente proporcionalidad entre peso y deformación. De la figura 9, vemos una relación aproximadamente proporcional entre esfuerzo y deformación unitaria por tracción si añadimos el origen a la gráfica.\\
3) A partir de los gráficos, determinar(si es que hay) la constante recuperadora del resorte y el módulo de Young. Si no es posible calcularlo, explicar como se debería calcular.\\
R: De la figura 8, se puede establecer la constante elástica a partir de la pendiente de la gráfica, cuyo valor es $56,832 \frac{\mathrm{~N}}{\mathrm{~m}}$. Respecto a la figura 9 , al haber una relación aproximadamente proporcional entre esfuerzo y deformación unitaria por tensión, podemos establecer el valor del módulo de young como la pendiente de la gráfica, cuyo valor es 155488 Pa.\\
4) A partir de los gráficos de la pregunta 2 , determinar por integración numérica el trabajo realizado para producir la deformación del resorte, desde su posición de equilibrio hasta la tercera carga.\\
R: Usando regla del trapecio para la gráfica presente en la figura 8 y asumiendo que la posición de equilibrio es el origen(debido a que hay una deformación inicia por el peso del resorte), entonces tenemos los siguientes resultados:

$$
\begin{aligned}
S & \approx \sum_{i=1}^{3} \frac{\left(F\left(x_{i}\right)+F\left(x_{i-1}\right)\right)\left(x_{i}-x_{i-1}\right)}{2} \\
& =0,7824129 \mathrm{~J} \\
& \approx 0,7824 \mathrm{~J}
\end{aligned}
$$

\begin{enumerate}
  \setcounter{enumi}{4}
  \item Para el caso de la liga o del jebe, llenar la tabla correspondiente dada por el profesor para procesos de carga y descarga. Luego, representar los datos obtenidos en la gráfica $\sigma$ vs $\varepsilon$. ¿Qué representa el área encerrada por esta curva?\\
R: Gráfica presente en la figura 10 de la página 15. Para el diagrama esfuerzo deformación de la liga, tenemos que el área bajo la curva tanto para carga y descarga\\
representa la tenacidad de la liga en carga y descarga, respectivamente; la tenacidad mide la capacidad del material para absorber energía sin romperse. Por lo que, el área encerrada por las dos curvas representaría la variación de tenacidad en la liga antes y después de estirarse.
  \item Determinar de forma aproximada el área encerrada por la curva de deformaciones respecto a la liga.\\
R: Aplicando la regla del trapecio nuevamente, tenemos:
\end{enumerate}

$$
\begin{aligned}
S_{\text {descarga }} & \approx \sum_{i=1}^{6} \frac{\left(\sigma_{\text {descarga }}\left(\varepsilon_{i}\right)+\sigma_{\text {descarga }}\left(\varepsilon_{i-1}\right)\right)\left(\varepsilon_{i}-\varepsilon_{i-1}\right)}{2} \\
& =50760,8646 P a \\
S_{\text {carga }} & \approx \sum_{i=1}^{6} \frac{\left(\sigma_{\text {carga }}\left(\varepsilon_{i}\right)+\sigma_{\text {carga }}\left(\varepsilon_{i-1}\right)\right)\left(\varepsilon_{i}-\varepsilon_{i-1}\right)}{2} \\
& =60392,9971 P a \\
S_{\text {carga }}-S_{\text {descarga }} & =S_{\text {encerrada }} \approx 9632,1325 P a
\end{aligned}
$$

Luego $S_{\text {encerrada }}$ sería la variación de la tenacidad respecto a carga y descarga.\\
7) Definir esfuerzo de fluencia, esfuerzo límite, módulo de elasticidad en tracción y compresión.\\
R: El esfuerzo de fluencia representa el máximo respecto al comportamiento elástico(un esfuerzo mayor ocasiona deformaciones permanentes).\\
El esfuerzo límite representa al máximo respecto al comportamiento plástico del objeto(un esfuerzo mayor ocasiona fracturas).\\
Para esfuerzos y deformaciones unitarias de menor magnitud, se establece una relación directamente proporcional(Ley de Hooke), cuya constante de proporcionalidad viene a ser el módulo de elasticidad(tanto para tracción como para compresión, como lo explicado en el fundamento teórico, páginas 3 y 4).\\
8) ¿Qué entiende por esfuerzo normal? Explique.\\
¿Existe diferencia entre un esfuerzo tangencial y un esfuerzo de torsión?\\
R: Respecto al esfuerzo normal, viene a ser aquel producido por una fuerza perpendicular a la sección transversal del objeto(produciendo tracción o compresión).\\
El esfuerzo tangencial es aquel producido por una fuerza paralela a la superficie transversal del objeto.\\
El esfuerzo de torsión se define como la capacidad para realizar torsión de objetos en rotación alrededor de un eje fijo.\\
Se observa una clara diferencia entre esfuerzo tangencial(o cortante) y esfuerzo de torsión. El esfuerzo tangencial no genera siempre una torsión respecto al objeto, mientras que el esfuerzo de torsión es justamente la capacidad de realizar torsión en un objeto, aplicando fuerzas.

\subsection{6.2. Observaciones}
\begin{itemize}
  \item El hecho de que solo se haya medido el diámetro de la sección transversal del resorte respecto a su mitad, no nos da una información clara respecto al esfuerzo real, debido a que las deformaciones transversales generan áreas transversales diferentes respecto a cada parte del resorte y a la liga.
  \item Respecto al diagrama esfuerzo y deformación unitaria por tracción en la liga(Figura 10), no se pudo realizar un ajuste polinomial debido a la relación entre las magnitudes respecto a los procesos de carga y descarga, por lo que se tuvo que trazar la curva a mano alzada.
  \item Durante el experimento, fue muy difícil realizar las mediciones correspondientes con el vernier, por lo que debe existir mayor incertidumbre respecto a las mediciones realizadas con esta herramienta.
  \item El hecho de que algunas de las deformaciones unitarias en la tabla 2 sean mayor a 1 , es porque se usó la longitud natural del resorte, y este superó tal longitud al poner más cargas.
  \item Respecto a la liga, esta tiene un comportamiento elástico lento respecto al tiempo transcurrido en procesos de carga y descarga.
  \item El hecho de que algunos de los esfuerzos presentes en las tablas 2,5 y 6 superen la presión atmosférica(en magnitud), es porque el área de la sección transversal del resorte y liga es mucho menor en comparación a la magnitud de la fuerza normal.
\end{itemize}

\subsection{6.3. Sugerencias}
\begin{itemize}
  \item El vernier digital es requerido para unas mediciones con menor incertidumbre y menor dificultad.
  \item Es recomendable reponer la liga, debido a que ya se estiró bastante por múltiples experimentos anteriores.
  \item Es recomendable reponer el resorte también, debido a que este parece estar algo oxidado, perturbando su composición y por ende, el módulo de young.
\end{itemize}

